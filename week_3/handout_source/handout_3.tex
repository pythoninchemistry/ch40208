\documentclass[a4paper]{article}
\usepackage{titling}
\usepackage{authblk}
\usepackage{fancyhdr}
\usepackage{hyperref}
\usepackage{rsc}
\usepackage{siunitx}
\usepackage{graphicx}
\usepackage{listings}
\usepackage{color}

\definecolor{dkgreen}{rgb}{0,0.6,0}
\definecolor{gray}{rgb}{0.5,0.5,0.5}
\definecolor{mauve}{rgb}{0.58,0,0.82}

\lstset{frame=tb,
  language=Python,
  aboveskip=3mm,
  belowskip=3mm,
  showstringspaces=false,
  columns=flexible,
  basicstyle={\ttfamily},
  numbers=none,
  numberstyle=\tiny\color{gray},
  keywordstyle=\color{blue},
  commentstyle=\color{dkgreen},
  stringstyle=\color{mauve},
  breaklines=true,
  breakatwhitespace=true,
  tabsize=3
}
\DeclareSIUnit\Fahrenheit{\degree F}

\title{Lecture 3: Lists, arrays, and optimisation with NumPy}
\author[1]{Dr Benjamin J. Morgan}
\author[1,2]{Dr Andrew R. McCluskey}
\affil[1]{Department of Chemistry, University of Bath, email: b.j.morgan@bath.ac.uk}
\affil[2]{Diamond Light Source, email: andrew.mccluskey@diamond.ac.uk}
\setcounter{Maxaffil}{0}
\renewcommand\Affilfont{\itshape\small}

\pagestyle{fancy}
\fancyhf{}
\rhead{CH40208}
\lhead{\thetitle}
\rfoot{\thepage}

\begin{document}
\maketitle

\section*{Aim}
This lecture will introduce lists, arrays, and show how the NumPy library can be used to make your code faster.

\section{Lists}
The Python programming language natively includes the ability to group together a series of objects.
These are \texttt{lists} and are one of the most powerful Python objects.
Lists are an ordered set of objects, from which it is possible to pick all, one, or many values.
A list is defined as follows,
\begin{lstlisting}
# Making a list

elements = ["Hydrogen", "Helium", "Lithium", "Beryllium", "Boron", "Carbon", "Nitrogen", "Oxygen"]
\end{lstlisting}
Having defined the list, it is then possible to select individual items of the list by using the following syntax,
\begin{lstlisting}
# Printing some items

print(elements[0], elements[4], elements[-1])
\end{lstlisting}
Note, that Python starts counting from the number $0$, and using the minus sign we can ask Python to count from the end.
This means that the above code should print, \texttt{"Hydrogen", "Boron", "Oxygen"}.
This counting from $0$ means that in the above list, the string \texttt{"Hydrogen"} would be referred to as the zeroth object in the list, while \texttt{"Helium"} would be the first.

In addition to making use of single objects from within a list, it is also possible to create sublists, for example,
\begin{lstlisting}
# Just the first 4 elements

print(elements[0:4])
\end{lstlisting}
Note that above, the numbers on either side of the colon the list indices.
However, rather strangely, the sublist created is \textbf{inclusive} of the first number and \textbf{exclusive} of the second.
Additionally, it is possible to select non-consecutive objects from a list by placing commas between the indices,
\begin{lstlisting}
# Just the gases

print(elements[0, 1, 6, 7])
\end{lstlisting}
The final point about \texttt{lists} is that the data that they hold does not all need to be the same time.
For example, the list below contains a \texttt{float}, two \texttt{str}, a \texttt{complex} number and an \texttt{int},
\begin{lstlisting}
# List of many types

a_new_list = ['hello', 12.41242, 5 + 8j, 'sadness', 2]
print(a_new_list)
\end{lstlisting}

\section{NumPy Arrays}

NumPy (or \texttt{numpy} or more commonly \texttt{np}) is a library that Python can use that is designed and optimised for doing numerical operations.\cite{numpy}
Over this course you will be introduced to many other Python libraries, in order to use any of these you must \texttt{import} them,
\begin{lstlisting}
# Import NumPy

import numpy as np
\end{lstlisting}
This asks the Python interperator to go and find the NumPy library, then in order to reduce the amount of typing (programmers are lazy), we give the library the alias \texttt{np}.

One of the most powerful features of the NumPy library is the \texttt{array}, these are similar to lists but with some important differences.



\bibliographystyle{rsc}
\bibliography{handout_3}

\end{document}
