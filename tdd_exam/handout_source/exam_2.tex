\documentclass[a4paper]{article}
\usepackage{titling}
\usepackage{authblk}
\usepackage{fancyhdr}
\usepackage[hyphens]{url}
\usepackage{hyperref}
\usepackage{rsc}
\usepackage{siunitx}
\usepackage{graphicx}
\usepackage{mhchem}
\usepackage{amsmath}
\usepackage{listings}
\usepackage{color}
\usepackage[htt]{hyphenat}
\usepackage{subcaption}

\definecolor{dkgreen}{rgb}{0,0.6,0}
\definecolor{gray}{rgb}{0.5,0.5,0.5}
\definecolor{mauve}{rgb}{0.58,0,0.82}
\graphicspath{{../figures/}}

\lstset{frame=tb,
  language=Python,
  aboveskip=3mm,
  belowskip=3mm,
  showstringspaces=false,
  columns=flexible,
  basicstyle={\ttfamily},
  numbers=none,
  numberstyle=\tiny\color{gray},
  keywordstyle=\color{blue},
  commentstyle=\color{dkgreen},
  stringstyle=\color{mauve},
  breaklines=true,
  breakatwhitespace=true,
  tabsize=3,
  postbreak=\mbox{\textcolor{red}{$\hookrightarrow$}\space},
  columns=fixed,basewidth=.5em,
}

% \setcounter{section}{-1}

\title{CH40208 Programming Assessment}
\author{Assessment 2}
% \author[1]{Dr Benjamin J. Morgan}
% \author[1,2]{Dr Andrew R. McCluskey}
% \affil[1]{Department of Chemistry, University of Bath, email: b.j.morgan@bath.ac.uk}
% \affil[2]{Diamond Light Source, email: andrew.mccluskey@diamond.ac.uk}
% \setcounter{Maxaffil}{0}
\renewcommand\Affilfont{\itshape\small}
\newcommand{\bvec}[1]{\boldsymbol{\mathbf{#1}}}
\newcommand{\norm}[1]{\left\lVert #1\right\rVert}
\newcommand{\cvec}[2]{\begin{bmatrix}#1\\#2\end{bmatrix}}
\newcommand{\tmatrix}[4]{\begin{bmatrix}#1&#2\\#3&#4\end{bmatrix}}

\pagestyle{fancy}
\fancyhf{}
\rhead{CH40208}
\lhead{\thetitle}
\rfoot{\thepage}
\date{11/12/19}

\begin{document}
\maketitle

\section*{Instructions}
\begin{enumerate}
  \item Read through all of the files and these instructions before you begin. 
  \item Before starting the assessment, ensure that you have downloaded the \texttt{functions.py} module and \texttt{tests.ipynb} notebook from Moodle. 
  \item The aim of this assessment is to follows the function design specifications described in the notebook and write functions (in the \texttt{functions.py}) module that adhere to the specification and pass the tests in the notebook. 
  \item Each of the five problems are weighted equally in the marking scheme; whcih includes a wide variety of aspects of programming.
  \item Remember, that to get the notebook to recognise changes in the \texttt{functions.py} module is it necessary to run \\\textbf{Kernel \texttt{=>} Restart}\\ from the menu. 
  \item \textbf{You will have up to 2 hours to work on this assessment}. The assessment is ``open book'' and you can refer to any notes you have from the course.
  \item Once the assessment begins, no internet use is permitted.\\\textbf{Students using the internet will receive a mark of 0}.
  \item When you have finished, or at the end of the assessment time, please upload your final \texttt{functions.py} to Moodle. Please run \\\textbf{Kernel \texttt{=>} Restart and Run All}\\ on the notebook and check your output. Be sure to upload the latest version of your module from your \texttt{H:} drive (check the modification time).
  \item Before uploading, please change the name of the module to \texttt{Firstname\_Surname.py} (note you probably shouldn't do this until the end as each of the tests assumes the functions to be imported from the \texttt{functions.py} module).
  \item You will be told when there are 15 minutes remaining.
  \item If you finish with more than 30 minutes remaining, please upload your notebook and leave quietly without disturbing the other students.
\end{enumerate}

\end{document}
