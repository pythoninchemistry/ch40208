\documentclass[a4paper]{article}
\usepackage{titling}
\usepackage{authblk}
\usepackage{fancyhdr}
\usepackage{hyperref}
\usepackage{rsc}


\title{Lecture 1: Introduction to Python}
\author[1]{Dr Benjamin J. Morgan}
\author[1,2]{Dr Andrew R. McCluskey}
\affil[1]{Department of Chemistry, University of Bath, email: b.j.morgan@bath.ac.uk}
\affil[2]{Diamond Light Source, email: andrew.mccluskey@diamond.ac.uk}
\setcounter{Maxaffil}{0}
\renewcommand\Affilfont{\itshape\small}

\pagestyle{fancy}
\fancyhf{}
\rhead{CH40208}
\lhead{\thetitle}

\begin{document}
\maketitle

\section*{Aim}
In this lecture, you will be introduced to Pythonic variable types, basic arithmetic, input and output (I/O) and intrinsic functions. 

\newpage
\section{Introduction}

The aim of this course is to develop skills in the user of computer programming (particularly in the Python programming language), building on the skills learned in the first and second year Computational Chemistry laboratory. 
You will then put these skills into practice, using Python to analyse chemical structures and perform quantum mechanical chemical calculations. 

The Python programming language is one of the most popular programming languages in the world, ranking third on the TIOBE index\footnote{The TIOBE index is an indicator of the popularity of a given programming language.} in June 2019\cite{tiobe_index}, with the largest rate of change. 
Additionally,

\bibliographystyle{rsc}
\bibliography{handout_1}

\end{document}