\documentclass[a4paper]{article}
\usepackage{titling}
\usepackage{authblk}
\usepackage{fancyhdr}
\usepackage[hyphens]{url}
\usepackage{hyperref}
\usepackage{rsc}
\usepackage{siunitx}
\usepackage{graphicx}
\usepackage{mhchem}
\usepackage{amsmath}
\usepackage{listings}
\usepackage{color}
\usepackage[htt]{hyphenat}
\usepackage{subcaption}

\definecolor{dkgreen}{rgb}{0,0.6,0}
\definecolor{gray}{rgb}{0.5,0.5,0.5}
\definecolor{mauve}{rgb}{0.58,0,0.82}
\graphicspath{{../figures/}}

\lstset{frame=tb,
  language=Python,
  aboveskip=3mm,
  belowskip=3mm,
  showstringspaces=false,
  columns=flexible,
  basicstyle={\ttfamily},
  numbers=none,
  numberstyle=\tiny\color{gray},
  keywordstyle=\color{blue},
  commentstyle=\color{dkgreen},
  stringstyle=\color{mauve},
  breaklines=true,
  breakatwhitespace=true,
  tabsize=3,
  postbreak=\mbox{\textcolor{red}{$\hookrightarrow$}\space},
  columns=fixed,basewidth=.5em,
}

% \setcounter{section}{-1}

\title{CH40208 Programming Assessment}
\author{Assessment 1}
% \author[1]{Dr Benjamin J. Morgan}
% \author[1,2]{Dr Andrew R. McCluskey}
% \affil[1]{Department of Chemistry, University of Bath, email: b.j.morgan@bath.ac.uk}
% \affil[2]{Diamond Light Source, email: andrew.mccluskey@diamond.ac.uk}
% \setcounter{Maxaffil}{0}
\renewcommand\Affilfont{\itshape\small}
\newcommand{\bvec}[1]{\boldsymbol{\mathbf{#1}}}
\newcommand{\norm}[1]{\left\lVert #1\right\rVert}
\newcommand{\cvec}[2]{\begin{bmatrix}#1\\#2\end{bmatrix}}
\newcommand{\tmatrix}[4]{\begin{bmatrix}#1&#2\\#3&#4\end{bmatrix}}

\pagestyle{fancy}
\fancyhf{}
\rhead{CH40208}
\lhead{\thetitle}
\rfoot{\thepage}
\date{4/12/19}

\begin{document}
\maketitle

\section*{Instructions}
\begin{enumerate}
  \item Before starting make sure that you have downloaded the \texttt{mo\_plotting} module, and that you can import the \texttt{plot\_linear\_mo} and \texttt{plot\_cyclic\_mo} functions. These functions are provided so that you can plot the molecular orbitals you will calculate during the assessment.

  Both functions take a \emph{list} of coefficients to define the p-orbitals that make up each molecular orbital:
  \begin{lstlisting}
  >>> from mo_plotting import plot_linear_mo
  >>> coefficients = [0.5, 0.5, 0.5, -0.5, -0.5, -0.5]
  >>> plot_linear_mo(coefficients)
  \end{lstlisting}


\begin{figure}[h!]
  \centering
  \resizebox{8cm}{!}{\includegraphics*{linear_mo_example.pdf}} %
    \label{fig:linear_mo_example}
\end{figure}
\begin{lstlisting}
  >>> from mo_plotting import plot_cyclic_mo
  >>> coefficients = [0.5, 0.5, 0.5, -0.5, -0.5, -0.5]
  >>> plot_cyclic_mo(coefficients)
  \end{lstlisting}
\begin{figure}[h!]
  \centering
  \resizebox{4cm}{!}{\includegraphics*{cyclic_mo_example.pdf}} %
    \label{fig:cyclic_mo_example}
\end{figure}
  \item Before starting, please read through this entire booklet. The first section provides the theoretical background for the assessment problems. The second section describes a set of \texttt{numpy} functions that you may find useful in completing this assessment (or not). The assessment problems are given in the third section. Reading through the full problem may help you to think about how to plan and organise your code.
  \item We advise periodically checking your notebook runs from the top, by running \\\textbf{Kernel \texttt{=>} Restart and Run All}\\ from the menu.
  \item \textbf{You will have up to 2.5 hours to work on the assessment}. The assessment is ``open book'' and you can refer to any notes you have from the course.
  \item Once the assessment begins, no internet use is permitted.\\\textbf{Students using the internet will receive a mark of 0}.
  \item When you have finished, or at the end of the assessment time, please upload your final notebook to Moodle. Before uploading please run \\\textbf{Kernel \texttt{=>} Restart and Run All}\\ on your notebook and check your output. Then save your notebook using \\\textbf{File \texttt{=>} Save and Checkpoint}\\ Be sure to upload the latest version of your notebook from your \texttt{H:} drive (check the modification time).
  \item You will be told when there are 15 minutes remaining.
  \item If you finish with more than 30 minutes remaining, please upload your notebook and leave quietly without disturbing the other students.
\end{enumerate}
\pagebreak
\section*{Background}
H\"{u}ckel molecular orbital theory is a simple theory for predicting the shapes and energies of molecular orbitals in conjugated hydrocarbons, such as ethene or benzene. In general molecular orbital theory, we constuct our molecular orbitals, $\bvec{\Psi}$, as a linear combination of atomic orbitals, $\bvec{\phi}$;
\begin{equation}
\bvec{\Psi} = \sum_i c_i\bvec{\phi}_i.
\label{eqn:LCAOs}
\end{equation}
To find the coefficients $c_i$ and energies for each molecular orbital we solve the secular equations
\begin{equation}
\bvec{H}\bvec{\Psi} = E\bvec{S}\bvec{\Psi}
\end{equation}
where $\bvec{H}$ is the Hamiltonian matrix, with the general form
\begin{equation}
\bvec{H}=\begin{bmatrix}
H_{11} & H_{12} & \ldots & H_{1N} \\
H_{21} & H_{22} & \ldots & H_{2N} \\
\vdots & \vdots & \ddots & \vdots \\
H_{N1} & H_{N2} & \ldots & H_{NN},
\end{bmatrix}
\end{equation}
where
\begin{equation}
H_{ij} = \int \phi_i \hat{H} \phi_j
\end{equation}

and $\bvec{S}$ is the overlap matrix, with the general form
\begin{equation}
\bvec{S}=\begin{bmatrix}
S_{11} & S_{12} & \ldots & S_{1N} \\
S_{21} & S_{22} & \ldots & S_{2N} \\
\vdots & \vdots & \ddots & \vdots \\
S_{N1} & S_{N2} & \ldots & S_{NN},
\end{bmatrix}
\end{equation}

where
\begin{equation}
S_{ij} = \int \phi_i \phi_j.
\end{equation}

The H\"{u}ckel model of conjugated $\pi$-systems makes a number of approximations that allow us to greatly simplify the secular equations:
\begin{enumerate}
  \item The atomic orbital basis only includes $p_z$ orbitals.
  \item The $p_z$ orbitals are orthonormal:
    \begin{equation}
    S_{ij} = \begin{cases}
        1 & \text{if}\ i=j, \\
        0 & \text{if}\ i\neq j.
      \end{cases}
    \end{equation}
    \item Terms in $\bvec{H}$ where $i=j$ correspond to the average energy of an electron in a carbon $p_z$ orbital:
    \begin{equation}
    H_{ii} = \alpha.
    \end{equation}
    \item Terms in $\bvec{H}$ are zero for any integrals that involve atoms $i$ and $j$ that are not adjacent. This describes the fact that $p_z$ orbitals on non-adjacent carbon atoms have very little overlap.
    \begin{equation}
    H_{ij} = \begin{cases}
      \beta & \text{if }i\text{ and }j\text{ are neighbours},\\
      0 & \text{otherwise}.
      \end{cases}
    \end{equation}
\end{enumerate}
These approximations convert our secular equations into an eigenvalue equation
\begin{equation}
\bvec{H}\bvec{\Psi} = E\bvec{\Psi},
\end{equation}
where for a \emph{linear molecule} the Hamiltonian matrix has values of $\alpha$ along the diagonal and values of $\beta$ on the superdiagonal and subdiagonal. e.g.\ for butadiene ($N=4$),
\begin{equation}
\bvec{H} = \begin{bmatrix}
\alpha & \beta  & 0      & 0 \\
\beta  & \alpha & \beta  & 0 \\
0      & \beta  & \alpha & \beta \\
0      & 0      & \beta  & \alpha.
\end{bmatrix}
\end{equation}
For a \emph{cyclic} molecule, atoms 1 and $N$ are adjacent, and we include additional terms in the top right and bottom left corners. e.g. for benzene ($N=6$),
\begin{equation}
\bvec{H} = \begin{bmatrix}
\alpha & \beta  & 0      & 0      & 0      & \beta \\
\beta  & \alpha & \beta  & 0      & 0      & 0     \\
0      & \beta  & \alpha & \beta  & 0      & 0     \\
0      & 0      & \beta  & \alpha & \beta  & 0     \\
0      & 0      & 0      & \beta  & \alpha & \beta \\
\beta  & 0      & 0      & 0      & \beta  & \alpha.
\end{bmatrix}
\end{equation}

Once we have the matrix $\bvec{H}$ we can compute the molecular orbitals, $\Phi_i$ and their corresponding energies $\left\{E_i\right\}$ by calculating the eigenvectors and eigenvalues of $H$. The eigenvectors of $\bvec{H}$ give coefficients $\left\{c_i\right\}$ in Equation \ref{eqn:LCAOs}, and the eigenvalues of $\bvec{H}$ are their energies.

The electronic energy of the $\pi$-electrons can be calculated by filling the molecular orbitals in order of increasing energy, with no more than two electrons in each orbital.

\pagebreak

\section{Useful \texttt{numpy} functions}
You may find it useful to use some or all of the following numpy functions in answering the question:

\begin{enumerate}
\item
  \texttt{numpy} arrays can be sorted using \texttt{np.sort()}
  \begin{lstlisting}
  >>> a = np.array([3, 4, 1, 7])
  >>> print(np.sort(a))
  [1, 3, 4, 7]
  \end{lstlisting}
  You can also sort a \texttt{numpy} \emph{in place} using \texttt{array\_name.sort()}
  \begin{lstlisting}
  >>> a = np.array([3, 4, 1 7])
  >>> a.sort()
  >>> print(a)
  [1, 3, 4, 7]
  \end{lstlisting}
\item
  The \texttt{np.roll()} function rotates the elements of an array:
    \begin{lstlisting}
    >>> a = np.arange(10)
    >>> print(a)
    [0 1 2 3 4 5 6 7 8 9]
    >>> b = np.roll(a,1)
    >>> print(b)
    [9 0 1 2 3 4 5 6 7 8]
    >>> c = np.roll(a, 3)
    >>> print(c)
    [7 8 9 0 1 2 3 4 5 6]
    \end{lstlisting}
  \item
    The \texttt{np.diag()} function creates a 2D \texttt{numpy} array, with an input array along one of the diagonals, and zeroes for all other elements:
    \begin{lstlisting}
    >>> a = np.diag([1.0, 1.0, 1.0])
    >>> print(a)
    [[1.0, 0.0, 0.0],
     [0.0, 1.0, 0.0],
     [0.0, 0.0, 1.0]]
    \end{lstlisting}
    You can specify diagonals above or below the main diagonal using the second optional argument:
    \begin{lstlisting}
    >>> a = np.diag([1.0, 1.0, 1.0], 1)
    >>> print(a)
    [[0.0, 1.0, 0.0, 0.0],
     [0.0, 0.0, 1.0, 0.0],
     [0.0, 0.0, 0.0, 1.0],
     [0.0, 0.0, 0.0, 0.0]]
    \end{lstlisting}
    Positive numbers specify diagonals above the main diagonal (as in the example above). Negative numbers specify diagonals below the main diagonal.
    \begin{lstlisting}
    >>> a = np.diag([1.0, 1.0, 1.0], -1)
    >>> print(a)
    [[0.0, 0.0, 0.0, 0.0],
     [1.0, 0.0, 0.0, 0.0],
     [0.0, 1.0, 0.0, 0.0],
     [0.0, 0.0, 1.0, 0.0]]
    \end{lstlisting}
\end{enumerate}

\pagebreak

\section*{Assessment}
In this assessment you will apply H\"{u}ckel theory to construct the Hamiltonian matrices for a series of molecules, and calculate the eigenvalues and eigenvectors to obtain the corresponding molecular orbitals and their energies.

Your working and calculations should be contained in a single Jupyter notebook, which you will upload to Moodle at the end of the assessment. Code should be written and arranged so that it is easily understandable.

\begin{enumerate}
  \item
  \begin{enumerate}
  \item Construct the $2\times2$ Hamiltonian matrix, $\bvec{H}$, for ethene (linear, $N=2$), using values of $\alpha=1.0$ and $\beta=-0.2$
  \item \label{ethene_energies}Calculate the eigenvalues and eigenvectors of $\bvec{H}$ using \texttt{np.linalg.eig()} (remembering that the eigenvectors are returned as columns of the second returned value).

  Show that there are two molecular orbitals with energies 
    \begin{eqnarray}
    E_1 & = & \alpha+\beta \\
    E_2 & = & \alpha-\beta
    \end{eqnarray}
  \item Using the \texttt{huckel\_vis.plot\_linear\_mo()} function, plot the two molecular orbitals in order of increasing energy. 
  \item Using the molecular orbital energies calculated in \ref{ethene_energies}, calculate the total electronic energy for this molecule. Remember there is one $\pi$-electron per p-orbital.
  \end{enumerate}
  \item
  \begin{enumerate}
  \item Construct the $3\times3$ Hamiltonian matrix, $\bvec{H}$ for cyclopropane (cyclic, $N=3$), using values of $\alpha=1.0$ and $\beta=-0.2$.
  \item \label{n3_energies}Calculate the eigenvalues and eigenvectors of $\bvec{H}$ using \texttt{np.linalg.eig()}. \\ Show that there are three molecular orbitals with energies 
    \begin{eqnarray}
    E_1 & = & \alpha+2\beta \\
    E_2 & = & \alpha-\beta \\
    E_3 & = & \alpha-\beta
    \end{eqnarray}
  \item Using the \texttt{plot\_cyclic\_mo()} function, plot the three molecular orbitals in order of increasing energy. 
  \item Using the molecular orbital energies calculated in \ref{n3_energies}, calculate the total electronic energy for this molecule.

  \end{enumerate}
  \item
  \begin{enumerate}
    \item \label{n4_energies}Construct the $4\times4$ Hamiltonian matrices for butadiene (linear, $N=4$) and for cyclobutadiene (cyclic, $N=4$), using values of $\alpha=1.0$ and $\beta=-0.2$. For each molecule, calculate the eigenvalues and eigenvectors, and plot the molecular orbitals in order of increasing energy. \\ To plot the molecular orbitals use the \texttt{plot\_linear\_mo()}  and \texttt{plot\_cyclic\_mo()} functions.
    \item Using the molecular orbital energies calculated in \ref{n4_energies}, calculate the total electronic energy for these molecules. Show that the cyclic form is \emph{unstable} with respect to the linear form.
  \end{enumerate}
  \item
  \begin{enumerate}
    \item \label{n6_energies}Construct the $6\times6$ Hamiltonian matrices for hexatriene (linear, $N=6$) and for benzene (cyclic, $N=6$), using values of $\alpha=1.0$ and $\beta=-0.2$. For each molecule, calculate the eigenvalues and eigenvectors, and plot the molecular orbitals in order of increasing energy.
    \item Using the molecular orbital energies calculated in \ref{n6_energies}, calculate the total electronic energy for these molecules. Show that the cyclic form is \emph{stable} with respect to the linear form.
  \end{enumerate}
  \item
  \begin{enumerate}
    \item By writing suitable functions (if you have not done so already), calculate the energy difference between linear and cyclic alkenes for $N=3$ to $N=20$, using values of $\alpha=1.0$ and $\beta=-0.2$.

    Plot the energy difference $E_\mathrm{cyclic}-E_\mathrm{linear}$ as a function of $N$. Your results should illustrate H\"{u}ckel's rule: that a cyclic, planar molecule is considered \emph{aromatic} if it has $4n+2$ $\pi$ electrons. 
  \end{enumerate}
\end{enumerate}

\end{document}
